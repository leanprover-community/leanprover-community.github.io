\documentclass[a4paper]{article}
\usepackage[T1]{fontenc}
\usepackage[utf8]{inputenc}
\usepackage[english]{babel}

%\usepackage[urw-garamond,expert,uppercase=upright,greeklowercase=upright]{mathdesign}
%\usepackage[osf,swashQ]{garamondx}
%
%\usepackage{microtype}

\usepackage[textwidth=17cm, textheight=27cm]{geometry}
\usepackage{booktabs, xspace}
\usepackage{amsmath,amsfonts,amssymb,longtable, fontawesome, hyperref}

\newcommand{\lean}[1]{{\tt #1}}
% \newcommand{\nv}{\textit{new\_name}\xspace}
% \newcommand{\nom}{\textit{name}\xspace}
\newcommand{\expr}[1][]{\textit{expr#1}\xspace}
\newcommand{\proposition}{\textit{prop}\xspace}
\newcommand{\tactic}[1][]{\textit{tac#1}\xspace} % also used for tactic sequences
\newcommand{\type}{\textit{type}\xspace}
\newcommand{\name}{\textit{name}\xspace}
\newcommand{\pattern}{\textit{pattern}\xspace}
\newcommand{\light}{\faLightbulbO\xspace}
\newcommand{\nat}{\textit{n}\xspace}
\newcommand{\internet}{\faGlobe\xspace}
\newcommand{\warning}{\faWarning\xspace}
\setlength\parindent{0pt}

% Original authors (Lean 3 version): Patrick Massot and Johan Commelin
% https://github.com/leanprover-community/lftcm2020/blob/master/lean-tactics.tex

\usepackage{makecell}
\usepackage{xcolor}

\begin{document}
\pagestyle{empty}
\begin{center}
 {\large\textsc{Lean 4 tactic cheatsheet}}

 {\scriptsize Last updated: \today}
\end{center}

\begin{center}
If a tactic is not recognized, write \lean{import Mathlib.Tactic} at the top of your file.\\
\smallskip

\setlength\tabcolsep{5mm}
\def\arraystretch{1.3}
\begin{tabular}{@{}lll@{}}
  \toprule
  Logical symbol & Appears in goal & Appears in hypothesis \\
  \midrule
  $\forall$ (for all) & \lean{intro x} & \lean{apply h} or \lean{specialize h x}  \\
  $\to$ (implies) & \lean{intro h} & \lean{apply h} or \lean{specialize h1 h2} \\
  $\neg$ (not) & \lean{intro h} & \lean{apply h} or \lean{contradiction}  \\
  $\leftrightarrow$ (if and only if)\qquad & \lean{constructor}  & \lean{rw [h]} or \lean{rw [← h]} or \lean{apply h.1} or \lean{apply h.2}\\
  $\wedge$ (and) & \lean{constructor} & \lean{obtain ⟨h1, h2⟩ := h} \\
  $\exists$ (there exists) & \lean{use x} & \lean{obtain ⟨x, hx⟩ := h} \\
  $\vee$ (or) & \lean{left} or \lean{right} & \lean{obtain h1 | h2 := h} \\
  $a = b$ (equality) & \lean{rfl} or \lean{ext} & \lean{rw [h]} or \lean{rw [← h]} \\
  \lean{True} & \lean{trivial} & --- \\
  \lean{False} & --- & \lean{contradiction} \\
\bottomrule
\end{tabular}
\end{center}

\begin{center}
\setlength\tabcolsep{5mm}
\def\arraystretch{1.3}
\begin{longtable}{@{}lp{113mm}@{}}
  \toprule
  Tactic & Effect \\
  \midrule
  &\textbf{Applying Lemmas}\\
  \lean{exact} \expr & prove the current goal exactly by \expr. \\
  \lean{apply} \expr & prove the current goal by applying \expr to some arguments. \\
  \lean{refine} \expr & like \lean{exact}, but \expr can contain \lean{?\_} that will be turned into a new goal. \\
  \lean{convert} \expr & prove the goal by showing that it is equal to the type of \expr. \\
  % assumption left out
  \hline
  &\textbf{Context manipulation}\\
  \lean{have h :} \proposition\ \lean{:=} \expr & add a new hypothesis \lean{h} of type \proposition. \warning \textbf{Do not use for data!} \\
  \lean{have h :} \proposition\ \lean{:= by} \tactic & add hypothesis \lean{h} after proving it using tactics. \warning \textbf{Do not use for data!} \\
  \lean{set x :} \type\ \lean{:=} \expr & add an abbreviation \lean{x} with value \expr. \\
  % let could be useful, but mostly superseded by set. let' allows underscores, but seems too niche to mention
  \lean{clear h} & remove hypothesis \lean{h} from the context.\\
  \lean{rename\_i x h} & rename the last inaccessible names with the given names.\\
  \lean{expose\_names} & make all inaccessible names accessible.\\
  \lean{change} \expr & replaces the goal by \expr, if they are equal by definition.\\
  \lean{generalize\_proofs} & add all proofs occurring in the goal to the local context.\\
  % show intentionally left out (duplicate of change)
  % rename, replace, revert
  \hline
  &\textbf{Rewriting and simplifying}\\
  \lean{rw [}\expr\lean{]} & in the goal, replace (all occurrences of) the left-hand side
  of \expr by its right-hand side. \expr must be an equality, iff statement or definition.\\
  \lean{rw [←}\expr\lean{]} & $\ldots$ rewrites using \expr from right-to-left. \\
  \lean{rw [}\expr\lean{] at h} & $\ldots$ rewrite in hypothesis \lean{h}. \\
  \lean{nth\_rw} \nat\ \lean{[}\expr\lean{]} & rewrite only the \nat-th occurrence of the rewrite rule \expr.\\
  \lean{grw [}\expr\lean{]} & Rewrite with inequalities or other relations. \\
  \lean{simp} & simplify the goal using all lemmas tagged \lean{@[simp]} and basic reductions. \\
  \lean{simp at h} & $\ldots$ simplify in hypothesis \lean{h}. \\
  \lean{simp [*, }\expr\lean{]} & $\ldots$ also simplify with all hypotheses and \expr. \\
  \lean{simp only [}\expr\lean{]}& $\ldots$ only simplify with \expr and basic reductions (not with simp-lemmas). \\
  \lean{simp?}& $\ldots$ let Lean speed up \lean{simp} by specifying which lemmas were used. \\
  \lean{simp\_rw [}\expr[1]\lean{, }\ldots\lean{]} & like \lean{rw}, but uses \lean{simp only} at each step. \\
  \lean{simp\_all} & repeatedly simplify the goal and all hypothesis using all hypotheses.\\
  % dsimp left out
  \lean{norm\_num} & simplify numerical expressions by calculating. \\
  \lean{push} \pattern\ / \lean{pull} \pattern & move the constant given by \pattern inwards / outwards.\\
  \lean{norm\_cast} & simplify the expression by moving casts ($\uparrow$) outwards.\\
  \lean{push\_cast} & push casts inwards.\\
  % apply_mod_cast, exact_mod_cast, rw_mod_cast, assumption_mod_cast left out
  \lean{conv =>} \textit{conv-tac} &
    \makecell[lt]{apply rewrite rules to only part of the goal.
    Use \lean{congr}, \lean{skip}, \lean{ext}, \\ \lean{lhs}, \lean{rhs}, \ldots to navigate to the desired subexpression.
    See \href{https://docs.lean-lang.org/theorem_proving_in_lean4/conv.html}{TPIL}.}\\
  \lean{split\_ifs} & case split on every occurrence of \lean{if} \textit{h} \lean{then} \expr\ \lean{else} \expr in the goal. \\
  % is this obsolete by split?
    \hline
  % (stronger than \lean{simp [*] at *})
  % \newpage
  &\textbf{Reasoning with equalities, inequalities, and other relations}\\
  % the technical difference is that the previous category can be applied to any subterm anywhere,
  % while the tactics below require that the goal is an (in)equality
  \makecell[lt]{\lean{calc} $a = b$ \lean{:= by} \tactic\\ \mbox{}\ \ $\_ \le c$ \lean{:= by} \tactic\\ \mbox{}\ \ $\_ < d$ \lean{:= by} \tactic} &
  \makecell[lt]{perform a calculation \\ \light after writing ``\lean{calc \_}'' Lean can generate a basic calc-block for you. \\
  \light after a \lean{by} shift-click on a subterm in the goal to create a new step.}\\
  \lean{rfl} & prove the current goal by reflexivity. \\
  \lean{symm} & swap a symmetric relation. \\
  \lean{trans} \expr & split a transitive relation into two parts with \expr in the middle. \\
  \lean{subst h} & if \lean{h} equates a variable with a value, substitute the value for the variable.\\
  \lean{ext} & prove an equality in a specified type (e.g. functions). \\
  \lean{apply\_fun} \expr\ \lean{at h} & apply \expr to both sides of the (in)equality \lean{h}. \\
  \lean{linear\_combination} & prove an equality by specifying it as a linear combination of hypotheses.\\
  % \lean{polyrith} & \internet query Sage to find a \lean{linear\_combination} invocation.\\
  \lean{congr} & prove an equality using congruence rules. \\
  % &\textbf{Reasoning with inequalities} (and other relations)\\
  \lean{gcongr} & prove an inequality using congruence rules. \\
  % mono obsoleted by gcongr?
  \lean{positivity} & prove goals of the form $0 < x$, $0 \le x$ and $x \ne 0$.\\
  \lean{bound} & prove inequalities based on the expression structure.\\
    % (overlaps in scope with \lean{gcongr} and \lean{positivity})
  \lean{order} & prove purely order-theoretic results.\\
  \lean{omega} / \lean{cutsat} & solve linear arithmetic problems over $\mathbb{N}$ or $\mathbb{Z}$. \\
  \lean{linarith} & prove linear (in)equalities from the hypotheses. \\
  \lean{nlinarith} & stronger variant of \lean{linarith} that can solve some nonlinear inequalities. \\
  \hline
  &\textbf{Reasoning techniques}\\
  \lean{exfalso} & replace the current goal by \lean{False}. \\
  \lean{by\_contra h} & proof by contradiction; adds the negation of the goal as hypothesis \lean{h}.\\
  % \lean{push\_neg} or \lean{push\_neg at h} & push negations into quantifiers and connectives in the goal (or in \lean{h}). % superseded by push/pull
  %; e.g.~change $\neg\;\forall$ \lean{x, P x} to $\exists$ \lean{x,} $\neg\;$\lean{P x}.
  % \\
  \lean{by\_cases h :} \proposition & case-split on \proposition. \\
  \makecell[lt]{\lean{induction n with }\\ \lean{| zero      =>} \tactic \\ \lean{| succ n ih =>} \tactic} &
  \makecell[lt]{prove a goal by induction on \lean{n}.\\ \\ \light after writing ``\lean{induction n}'' Lean can generate the cases for you.} \\
  \lean{choose f h using} \expr & extract a function from a forall-exists statement \expr.\\
  \lean{lift n to} \type\ \lean{using h} & lifts a variable to \type (e.g. $\mathbb{N}$) using side-condition \lean{h}.\\
  \lean{zify} / \lean{qify} / \lean{rify} & shift an (in)equality to $\mathbb{Z}$ / $\mathbb{Q}$ / $\mathbb{R}$. \\
  %maybe: classical
  % contrapose, absurd intentionally left out
  \hline
  &\textbf{Searching}\\
  \lean{exact?} & search for a single lemma that closes the goal using the current hypotheses. \\
  \lean{apply?} & gives a list of lemmas that can apply to the current goal. \\
  \lean{rw??} & allows you to shift-click on subterms to get rewrite suggestions. \\
  \lean{rw?} & gives a list of lemmas that can be used to rewrite the current goal. \\
  \lean{have? using h1, h2} & try to find facts that can be concluded by using both \lean{h1} and \lean{h2}. \\
  \lean{hint} & run a few common tactics on the goal, reporting which one succeeded. \\
  \hline
  &\textbf{General automation}\\
  \lean{grind} & general-purpose automation tool with many features. \\
  \makecell[lt]{\lean{ring} / \lean{noncomm\_ring} / \lean{module} \\ \lean{field} / \lean{abel} / \lean{group}} & prove the goal by using the axioms of a commutative ring / ring / module / field / abelian group / group. \\
  \lean{aesop} & simplify the goal, and use various techniques to prove the goal. \\
  \lean{tauto} & prove logical tautologies. \\
  \lean{decide} & run a decision procedure to prove the goal (if it is decidable). \\
  % \lean{slim\_check} & try to find a counterexample to the current goal.\\
  \hline
  % &\textbf{Goal manipulation}\\
  % \hline
  &\textbf{Operations on goals/tactics}\\
  \lean{swap} & swap the first two goals. \\
% pick_goal intentionally left out (too similar to on_goal)
  \lean{on\_goal} \nat \lean{=>} \tactic & run \tactic on goal \nat. \\
  \lean{all\_goals} \tactic & run \tactic on all goals. \\
  \lean{try} \tactic & run \tactic only if it succeeds. \\
  \tactic[1]\lean{;} \tactic[2] & run \tactic[1] and then \tactic[2] (same as putting them on separate lines). \\
  \tactic[1] \lean{<;>} \tactic[2] & run \tactic[1] and then \tactic[2] on all goals generated by \tactic[1]. \\
  \textcolor{red}{\lean{sorry}} & admit the current goal.\\
  \hline
  &\textbf{Domain-specific tactics}\\
  % finiteness
  \lean{fin\_cases h} & split a hypothesis \lean{h} into finitely many cases. \\
  \lean{interval\_cases n} & if split the goal into cases for each of the possible values for \lean{n}.\\
  % algebra
  \lean{compute\_degree} & prove (in)equalities about the degree of a polynomial \\
  \lean{monicity} & prove that a polynomial is monic \\
  % topology/analysis
  \lean{fun\_prop} & prove that a function satisfies a property (continuity, measurability, \ldots). \\
  \lean{measurability} & prove that a set or function is measurable. \\
  \lean{filter\_upwards [h1, h2]} & Show that an \lean{Eventually} goal follows from the given hypotheses. \\
  \lean{slice\_lhs}, \lean{slice\_rhs} & Focus on a part of a composition in a category.\\
    & See \href{https://github.com/leanprover-community/mathlib4/tree/master/Mathlib/Tactic/CategoryTheory}{the source code} for some other category theory tactics. \\
  % continuity intentionally left out
  % maybe mention bicategory, monoidal, mod_cases, reduce_mod_char
  % low-level control tactics intentionally left out: beta, delta,
  \bottomrule
\end{longtable}
\mbox{}\\
% maybe: generalize_proofs, infer_instance, inhabit, nontriviality, peel, move_add
% honourable mentions: \lean{zify}, qify,
\end{center}

\textbf{Usage note}

This is a quick overview of the most common tactics in Lean with only a short description. To learn more about a tactic and to learn its precise syntax or variants, consult its docstring or use \lean{\#help tactic} \tactic.

This list is not complete, and various tactics are intentionally left out.
\bigskip

\textbf{Some useful commands} (Some of these also work as tactics)

\begin{longtable}{@{}lp{113mm}@{}}
\lean{\#loogle} \textit{query} & \internet use \href{https://loogle.lean-lang.org/}{Loogle!} to find declarations. \\
\lean{\#leansearch "}\textit{query}\lean{."} & \internet use \href{https://leansearch.net}{LeanSearch} to find declarations. \\
\lean{\#exit} & don't compile code after this command.\\
\lean{\#lint} & run linters to find common mistakes in the code \emph{above} this command.\\
\lean{\#where} & print current opened namespaces, universes, variables and options.\\
\lean{\#min\_imports} & print the minimal imports needed for what you've done so far.\\
\lean{\#find\_home} \name & find a file that imports all prerequisites of \name.\\
\lean{\#help tactic} \tactic & find information about \tactic.\\
\lean{\#help} \textit{category} & list all tactics/commands/attributes/options/notations.\\
\end{longtable}

\textbf{Legend}\\
\begin{tabular}{ll}
\light & describes a code action for this tactic.\\
\internet & requires internet access.\\
\end{tabular}
\bigskip

% Do we want a list of commands?
% Very common commands: import, open, variable, namespace, def, theorem, example,
% Commands taught early: #check, #eval, #print
% Somewhat useful: whatsnew in, #synth, #exit, #min_imports, #find_home, #where
% Not useful to explain here IMO: #conv, #push_neg, #simp, #find,
% Only for advanced users: #find_home, #reduce, #redundant_imports

\end{document}
